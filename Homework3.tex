\documentclass[journal]{IEEEtran}
\usepackage{fancyvrb}


\begin{document}

\title{CPE 480 Assignment 2\\Multi-Cycle LOGICK}

\author{
Mark Noblin, Tanner Wilkerson, CJ Vanderpool
}
\maketitle

\begin{abstract}
The goal of this assignment was to utilize the assembly code written in a previous assignment to develop and produce a multi-cycle processor design. The multi-cycle design should be able to execute most of the commands encoded with in the assembly using Verilog programming as a source for logic.
\end{abstract}

\section{GENERAL APPROACH}
\subsection{Assignment Specifications}
	The processor to be developed in this assignment needs to take care of executing a specific set of commands from the assembly encoding. Our group decided to use a fresh encoding that was more tailored to the needs of the assignment. The commands that this assignment focuses on executing are those that do not require the Logarithmic Number System, or LNS, that will be implemented later on. This means that the processor should be able to execute ad, an, or, no, eo, mi, sr, br, jr, li, si, co, lo, st, and sy. The encoding also has a macro developed for the la command, which was discussed in the last assignment. This operation, however, does not require anything from our processor as it simply tells the processor the proper combination of li and si to perform.\\
	
	The advice given by the assignment is to stick to a top-down approach to designing the processor. Our group chose to do so, while also sticking to the ideology of breaking up our code chunks into seperate modules. This allows us to specify as much as possible what exactly verilog builds when the entire program is synthesized. It helps to avoid unecessary or messy synthesized blocks and helps in easing the stress of testing.\\
	
	Our design makes use of a few very important structure blocks to achieve the needed functionality. We have an ALU that is capable of handling those operations that specifically require arithmetic to be performed. We also have a mainmem and regfile block that can not only read and write but perform shifts to the left and right as well. We a block currently being called the oracle that specifies the structure of the Finite State Machine, or FSM, used to determine the execution of instructions. Lastly, we have a few smaller blocks, such as the MDR, MAR, A, Y, and condition blocks that help in the movement and execution of data onto our bus.\\
	
	The last step after designing all of this is our testing method. The first step in testing was to develop small code blocks capable of testing the logic of each of our instructions upon their execution using the operations we have available. For example, we might use ad to add two numbers and store them into memory, then use co to compare this to a predetermined result and see if this is correct. The testing program then would use a br operation to avoid a sy call that would halt the program if that result was incorrect. This provides a streamlined method for determining which operations function properly and which need work.

\subsection{Code Implementation}
	Our code implementation for this project is roughly 400 lines, so for the sake of brevity, some examples will be given and important code will be addressed in this section.\\
	
	The first piece of code to consider is the ALU. This unit is the heart of the processor, as most operations that can be performed are done so using operations built into the ALU. In looking at our encoding, the following operations are those that are specifically made to work with our ALU:
	
	\begin{Verbatim}[fontsize=\small]
	ad $.d, $.s, $.t := 0:4 .d:4 .s:4 .t:4
	an $.d, $.s, $.t := 1:4 .d:4 .s:4 .t:4
	or $.d, $.s, $.t := 2:4 .d:4 .s:4 .t:4
	no $.d, $.s := 3:4 .d:4 .s:4 0:4
	eo $.d, $.s, $.t := 4:4 .d:4 .s:4 .t:4
	\end{Verbatim}
	
	These five operations are defined within a case statement inside the ALU. The case statement is controlled by an input reg called c. The ALU receives this from the oracle module during execution of specific operations. The ALU operates on operands a and b, which are both given as inputs as well. Finally, after the ALU has performed calculation, it gives 2 outputs. One is of one word size, called out, that is the result of the calculation. The other is an 8 bit output of the condition result, which determines the state that the condition register should now be in based on the result of the calculation. This condition output is always calculated and returned in the event that it is needed at any point by the next operations. The code for the ALU in its entirety is as follows:
	
	\begin{Verbatim}[fontsize=\small]
	module alu(out, a, b, c, cond);
	  input `word a, b;
	  input [2:0] c;
	  output reg `word out;
	  output reg [7:0] cond;
	
	  always@ (c) begin
	    case(c)
		  0: out = a + b;
		  1: out = a & b;
		  2: out = a | b;
		  3: out = ~b;
		  4: out = a ^ b;
	    endcase
	  end
	
	  always@ (a, b)begin
	    if (a>b) cond = 8'b00001111;
	    else if (b>a) cond = 8'b01101001;
	    else	cond = 00010001;
	  end
	endmodule
	\end{Verbatim}
	
	The next piece of code to consider is the Regfile Module. This module is going to hold the needed register values to be passed onto the bus when operations call for them. The module for this regfile is capable of taking in a clk variable for synchronization purposes, an in value for data, a control value, and a sel value for selecting one of the available registers. It only has one output for passing back out a fetched register value. The Regfile is capable of not only reading and writing values, but shifting the values left and right as well. These operations are all performed through a case statement that comprises most of the body of the module. The code for this Regfile module can be seen as follows:
	
	\begin{Verbatim}[fontsize=\small]
	module regfile(clk, out, in, c, sel);
	  input clk;
	  input `word in;
	  input [3:0] c, sel; 
	  output `word out; 
	  reg `word r `regsize;
	
	  always@ (posedge clk)begin
		case(regc)
		0: r[sel] <= r[sel];
		1: r[sel] <= in;
		2: r[sel] <= r[sel] << 1;
		3: r[sel] <= r[sel] >> 1;
	  endcase
	  end
	  assign out = r[sel];
	endmodule
	\end{Verbatim}
	
	The Mainmem module is very similar to the Regfile module seen above. It lacks a few of the fancier feature that the Regfile has as it does not have a use for them. Instead, the Mainmem only has the ability to read and write to and from its contents, with a single control input to determine which of the two to do. the code for the Mainmem module can be seen as follows:
	
	\begin{Verbatim}[fontsize=\small]
	module mainMem(clk, write, datain,
	dataout, addr);
	  input clk, write;
	  input `word datain, addr;
	  output `word dataout;
	
	  reg `word memory `memsize;
	
	  always@ (posedge clk) begin
		if (write) memory[addr] <= datain;
		dataout <= memory[addr];
	  end
	
    endmodule
	\end{Verbatim}
	
	Next, the load\--store\--shift registers, or lss\_reg module is defined. This module aims to build all of the registers that aren't included in the Regfile module. It behaves similarly to the Regfile as well, including the shifting ability as well. It has an in and out for data movement, as well as an input c for control. The code for the lss\_reg is as follows:
	
	\begin{Verbatim}[fontsize=\small]
	module lss_reg(clk, out, in, c)
	  input [1:0] c;
	  input `word in;
	  output `word out;
	
	  always@ (posedge clk)begin
	    case(c)
		0: out <= out;
		1: out <= in;
		2: out <= out << 1;
		3: out <= out >> 1;
		endcase
	  end
	endmodule
	\end{Verbatim}
	
	The Oracle, in this implementation, is a large FSM whose outputs are used to control the rest of the processor. To implement such a large statemachine, a local parameter was used to enumerate each of the individual states. The fisrt state of each instruction will enumerated as its opcode, and states that follow will be enumerated in acending order starting at 50. 50 was chosen to allow for more instructions to be added since the log number system is yet to be implemented. This enumeration scheme allows for a minimal amount of case statements in the verilog code, i.e. any state that simply goes to the next numbered state can be combined into one case in the block. 
	
	The oracle is made from 3 always blocks. The first simply drives the module from the clock and implements a reset signal. The next always block defines the next state logic for every state used in this implementation. The final block drives the control signals that go directly to the other modules in this project. Most of the complexity in this project came from designing and implementing this module. Because of this, it was expected to be the most difficult to test.
	
	The top module in this
	
\section{Testing and Issues}
\subsection{Testing}
	Our testing is split up into a few different methods in order to ensure as complete of a testing routine as we can get. In this section, we'll step through all of these methods briefly and explain the general practice for testing using these methods. 
	
	The first method used is contained in the $Testing.txt$ file. This file contains multiple small blocks that aim to test the logic correctness of the operations defined in our assembly code. Each one has its own derived process for testing, however many of the operations have similar structure. For example, many of the operations tested from the ALU will place test operands into the operation to be tested, get a result from the operation, and use compare to see if the result matches a predetermined correct answer that will be stored in memory.
	
	In the second method, we utilized Vivado to ensure that the structure of the processor was correct. We were able to synthesize the design properly and view the structure that Vivado generated. In doing so, we got the correct structure for most if not all of our processor. This portion of testing did not necessarily allow for us to ensure that the logic of the processor is correct, but we were at least able to see that all of the individual modules described above were connected properly and that, assuming these modules were internally correct, the processor should work. Unfortunately, as seen in the $Known Issues$ section below, the modules were not internaly correct on all counts.
	
	The last method for testing was to synthesize the circuit as described above and use a dumpfile to observe the changing of signals within our design. This was performed in Vivado, which allowed us to select which specific signals we wanted to observe. We focused on the signals involved with the processor module so we could observe the steps taken inside the state machine for logic. This is where we uncovered some of the issues that are currently causing our processor not to function. 

\subsection{Known Issues}
	Our biggest known issue at the moment is that our processor does not work. We currently have a very large portion of the code done, but there is no promise that it functions currently. It is almost certain that the problem that is causing our code not to work properly comes from the complexity of the state machine. While we feel confident that the logic of our individual operation states is very close to correct, we have issues in the proper toggling of certain signals within the state machine that make it impossible for the state machine to advance through the necessary steps for an operation properly. Namely, the halt signal sent to the state machine is persistently turned on. 

\end{document}
